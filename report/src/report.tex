\documentclass{article}

%%% IMPORTS %%%
%\usepackage[welsh]{babel}
\usepackage{hyperref}
\usepackage{authblk}
\usepackage{float}
\usepackage{graphicx}
\usepackage[utf8]{inputenc}
\graphicspath{{img/}}


\author[1,3]{Preben Vangberg}
\author[2]{Sondre L. Gjesdal}
\author[3]{Håvard Syslak}

\affil[1]{Department of Computer Science, Aberystwyth University, Wales}
\affil[2]{Western Norway University of Applied Sciences, Norway}
\affil[3]{MMP PM MON MAIN AUT, Equinor ASA, Norway}


%%% DATA %%%
\title{Hacathon Hanes Report and Findings}

%%% DOCUMENT %%
\begin{document}

\maketitle
\newpage

\section{Introduction}

Hacathon Hanes is a hackathon set to be arranged by the National Library of Wales in Aberystwyth Tuesday the 11th of February.
Because of the inability of me (Preben Vangberg) to attend the hackathon in person, I got together with some friends during Saturday the 8th of February to organise a small local version of the Hackathon in Norway.
This report outlines the work we did and shows of some of the findings we were able to make during the session on Saturday.

Of the availible datasets, we decided to use Aberystwyth shipping data made available by the National Library.
We went through, filtered, and classified some of the data, in order to produce some results that were interesting to us.

\section{Classification and Filtering of Data}

The field "cause\_of\_leaving" was of some interest to us. The problem we had was that there was approximatlly 87.000 records, and almost as many causes for leaving. 
We therefore spent a significant amount of time classifing the differnent reasons for leaving. These were seperated into the following categories: remains on ship, discharged, shipwreck, illness, drowned, deceased, juristical reasons, deserted, transferred, hospitalised, abandoned, ship laid up, injured, and other. 

These classifications are not perfect and because data is missing, some data is hard to classify, and the semi-manual methodology used, it means that these classifications are approximates. However they still give some insight into the reasons why people left the ship.

The fields "joined\_ship\_date" and "date\_left", was also of some interest to us.
The issue we had with these fields was that not all fields contained dates, and those that did, not all of the dates where valid (Such as 31. of June).
We therefore wrote a short Python script that filters out entries that contains both a valid "joined\_ship\_date" and "date\_left". This means that a lot of entries which manualy could have been fixed, but wasn't. However we managed to get a total of 56842 entires, or around 65\% of the total, which we deemed good enough for our purpose.

It should be noted that we would have wanted to explore different methods and filters, but this is something that can be done on a later date.

\section{Results, Graphs, and Findings}

\subsection{Age Distribution}

Based on the graph presented here, we see that the age onboard of the ships are quite diverse. The youngest in the dataset was 10, and the oldest was 78 at the time of joining the ship. The curve seems to follow a normal distrubution peaking around 20 years old.

\subsection{Classification of Causes of Leaving}



\subsection{Recruitment, Dischargement, and Time Onboard}

This graph shows which months people joined the different ships and voyages. As you might can see, there is a significant increase in January and July, but there is still some spread evenly throughout the year.

This graph shows which months people left the ship. There seems to be trend indicating that people return mostly around the end of December/early January and late June/early July.

When we merge these together we can see an interesting picture:

This seems to indicate that the a lot or most of the crew is discharged in bulks, and hired shortly afterwards. This seems to happen in January and July with 6 months difference. When we consider the time spent onboard we can also see a spike around 180 days. There is also a spike around 70-90 days. This seems to indikate that a lot of ships tended to be out for 2 to 3 or 6 months at a time, then replacing the crew afterwards.
If we compare this to the shipping-industry today, we can see that this is the case today as well.
A lot of ships today discharges most of their crew and re-hire in set periods (every 6 months for example).


\section{Final Thoughs and Conclusions}

Even with the short amount of time, very basic methodolohies utilized, and overall simplicity of the resulting data, it was exciting, interesting and educational to work on this data. 
The results were interesting and tell us a bit about the life of seamen in the 19th centuary.
It would be interesting to see what could be achived with greater amount of time, and more sophisticated methods of classification and filtering could have produced, and it might be something that can be worth coming back to at a later date.

We would like to thank the National Library of Wales for arranging the hackathon even though we were unable to be there in person.
We had a lot of fun and we learnt a lot of new technologies and libraries which the authors hadn't had much experience in before.
We are definitly looking forward to any other hackathon arranged by the National Library, and we hope to be able to attend these in person.

\end{document}
